%documentclass[11pt]{seminar}
\documentclass[portrait,11pt]{seminar}

\slidefontsizes{10}

\newcommand\bs{\begin{slide*}}
\newcommand\es{\end{slide*}}

\newcommand\bi{\begin{myitemize}}
\newcommand\ei{\end{myitemize}}


\usepackage{ulem}
\usepackage{amsmath,amssymb,amsfonts,amsthm,graphicx}

\usepackage{color,semcolor}
\definecolor{green}{rgb}{0,0.8,0.2}

\newcommand\prob{\mathbb{P}}
\newcommand\E{\mathbb{E}}
\newcommand\SampleSpace{\mathbb{S}}
\newcommand\R{\mathbb{R}}
\newcommand\Z{\mathbb{Z}}
\newcommand\Var{\mathrm{Var}}
\newcommand\Cov{\mathrm{Cov}}
%\newcommand\mydefinition[1]{{\ \uwave{#1}}}
%\newcommand\mydefinition[1]{{\red \textbf{#1}}}
\newcommand\mydefinition[1]{{\textbf{#1}}}
\newcommand\mymath{\blue }
%\newcommand\myproof{\underline{Proof:} }
\newcommand\myproof{{Proof:} }
\newcommand\equals{{=}\,}
\newcommand\given{{\, | \,}}

\newcommand\hd[1]{\centerline{\large\bf #1}}
\newcommand\shd[1]{\underline{\large #1}}

\slideframe{none}
\newenvironment {myitemize} {
                 \begin{list}{$\bullet$ \hfill}
                 {\setlength{\labelwidth}{0.3 cm}
                  %\setlength{\leftmargin}{0em}
                  \setlength{\leftmargin}{0.15cm}
                  \setlength{\itemindent}{0.15cm}
                  \setlength{\labelsep}{0cm}
                  \setlength{\parsep}{0.2 ex}
%                  \setlength{\itemsep}{0.25 cm}
                  \setlength{\itemsep}{0.15 cm}
      \setlength{\topsep}{0.1cm}}} %space between title and 1st item
   {\end{list}}

\newenvironment {myequation} {\vspace{-1mm}\begin{equation*}}{\end{equation*}\vspace{1mm}}

\newenvironment {myeqnarray} {\vspace{-1mm}\mymath \begin{eqnarray*}}{\end{eqnarray*}\vspace{-1mm}}




\begin{document}
\bs
{\bf Building and maintaining healthy mentor/mentee relationships
}

\medskip

  What roles do mentorship relationships play in professional development of PhD students? 

% 2014 \it  ``The mentorship relationship partially decides how PhD students will move along the trajectory of their professional development.''

{\it ``Mentorship relationships are quite essential to the professional development of PhD students since mentors can suggest potential research directions, give guidance on what can be done to get positive results in research, offer encouragement when the beginning researcher has a hard time and help the mentee throughout his/her research career.''}

 \es \bs

  What do the mentee and mentor gain from the relationship?

% 2013 \it ``Mentors gain new ideas from a fresh perspective of their students, and also gain people that helps them in researching. Mentees gain a guide in learning how to conduct in research, and also a connection to the academic world.''

% 2014 ``By maintaining this mentorship relationship, the mentee and mentor can build a social cohesion, which can enable for them to create a strong research program and maintain a good friendship and relationship.''

{\it ``The mentee can have an assistance being a mature member of the scientific research group. He/she can get valuable information or mental support. The mentor can get a respect from the beginning researcher, and he/she can also get an new idea while discussing with the mentee.''}


 \es 

\bs

  Describe a situation in which the interests of the mentor and mentee are aligned. [In this context, ``interests'' means career or financial advancement.]

\medskip

{\it ``If a mentor and mentee have similar research interests or working styles.''}

\medskip

COMMENT ON WHAT YOU COULD LEARN FROM THIS RESPONSE ABOUT THE WORTHWHILENESS (OR LACK OF IT) OF RCRS EDUCATION.

\es

\bs

  Describe a situation in which the interests of the mentor and mentee are aligned. [In this context, ``interests'' means career or financial advancement.]


\medskip

{\it ``There are some mentors who want their mentees to get an academic career. If mentee also wants to become a professor, then their interests are aligned.''}


\medskip

THIS IS A REAL ISSUE. FROM AN RCRS PERSPECTIVE, IT IS ALSO NECESSARY TO DISCUSS THE USE OF THE WORD `INTERESTS' IN THIS CONTEXT, WHICH I THINK HERE IS PARTLY CORRECT BUT POSSIBLY PARTLY MISUNDERSTOOD. PLEASE COMMENT.

% 2014 \it ``Mentor and Mentee may be both working on the same project and success would ensure that both reap rewards.''

% 2014 ``For example, a mentor got a project with funding. The mentee worked with the mentor on the project. If they worked well, they would both gain academic progress and financial reward.''

% 2013  ``When a mentor and a mentee collaborate on a paper or a project, both of the individuals have a strong interest in having the paper published in a reputable journal and in having the project succeed in meeting its goals. The mentor also stands to gain doubly when their mentee is recognized, because their value as a mentor may be increased in the minds of other students and increase the pool and caliber of potential mentees.''


\es 


%\bs
%
%{\it ``One common situation is working on one thesis together.''}
%
%At the risk of over-interpreting this response...
%
%(i) What are the main things an advisor gains from advising a thesis?
%
%(ii) What are the main things a student gains from writing a thesis?
%
%(iii) When do these interests coincide? When do they conflict?
%
%\es

%\bs
%{\it ``The interests of mentor and mentee are aligned when mentee has the career planning in an area where the mentor has fairly good reputation and academic/social network.''}
%
%How do you balance the benefits (and disadvantages) of taking the opportunity to follow your own research interests with the benefits (and disadvantages) of staying close to your adviser's interests?
%
%\es

\bs

  Describe a situation in which the interests of the mentor and mentee are conflicting. 

{\it ``One typical example of conflicts would be the concept of intellectual property. For example, the mentor has a brilliant idea about a problem. The mentee is a senior in
the lab, and he/she is looking for a research fellow career in a competitive lab. In a job interview, which has not been confirmed with the mentor, the mentee discusses the mentor’s idea to the competitor. When the mentor realizes that his/her competitor knows about his idea, he gets really upset and disappointed from his mentee.''}


% 2014 \it ``Suppose the mentee is qualified to graduate and he wants to graduate. However, his mentor wants to keep him for another year to help with his own research. That’s one occasion conflicts of interest may arise.''

% 2014 ``For example, the mentor and the mentee co-authored a journal paper. They had conflicts on who would be the first-author.''

% 2013  ``The advisor might think having the student staying in the research group longer might be good for the whole research project, while the student might be eager to graduate.''

% 2013 ``A conflict might arise if the mentee thinks that by using more sophisticated equipment, the experiment can be performed more eciently, but the mentor is unwilling to use more funding and wants to keep the experiment simple.''

% 2013 ``The conflict might happen when the mentor and mentee have different ideas over a certain problem and none of them is ready to yield.''
 \es \bs
% 2014 {\it ``A professor has hired on a student as a research assistant, however, as the student has begun to further explore the area of research, the student’s interests have diverged from the professor’s interests. Now the student strongly desires exploring this new area of research for his/her dissertation while the professor, who wants to promote his/her own research, needs the funded student’s assistance.''}

\es

\bs

  Describe a situation in which the interests of the mentor and mentee are 
conflicting. 

{\it ``Mentor has been leading a group of researchers which consists of his/her mentees to work a few closely related problems. They have made good progress. The mentor wishes to publish a paper that presents the entire work which would make a big impact. However each of his mentees would want to publish a seperate (salami) paper so that their names can be all listed as first authors in the papers.''}

\medskip
IS THIS A ``CONFLICT OF INTEREST'' OR SIMPLY A DIFFERENCE OF OPINION? A SUBTLE QUESTION, ON WHICH ONE MIGHT TAKE EITHER VIEW...

\es

\bs

  How are mentorship relationships initiated? E.g., how do you find a thesis adviser?

{
\it  ``Generally , it is the mentee’s responsibility to look for the appropriate thesis adviser, and it might prove to be extremely helpful to talk to other senior PhD students about respective work interests and try to sort out who could be best suited in that regard as an adviser.''
}


{\it ``A student might find a thesis adviser by looking into faculty members’ research, reading recent papers, etc., and finding a professor who does work aligned with the student’s interests. A conversation with that faculty member may start to build a relationship between the two.''}


% 2014 { \it ``It's said in the article that usually potential mentors reach out to students to initiate a mentorship relationship. However, I think it’s more often the student’s job to talk to potential advisers and see if they are willing to mentor you.''}


\es
\bs

How common is it in Statistics for a faculty member to actively recruit a student, rather than waiting for potential PhD students to contact them?

\medskip

When might a faculty member initiate a mentorship relationship in a Statistics department? Compare this to other scientific fields.

% 2013 ``The student must decide whether he would like to get an exposure to new problems which he can solve by his own, or someone to thoroughly guide him in his research career. One should browse through the research papers of the desired mentor, try to look for mentors with projects at hand which appeal to the interests of the student.''

\es

%\bs

%From thesis adviser to mentor... 

%\it ``It is the same thing in any personal relationship, consistency and integrity over an extended period are usually required to establish the deep connection of a mentor/mentee relationship.''

%\es\bs

%\it ``To get familiar with the project the potential mentors are doing or will do and check own interest. Also, ask about the potential mentors’ mentoring style as well as what role I can play under his/her mentoring.''


%\es

\bs

  Collaboration: What are the advantages and disadvantages of building a mentorship relationship with a researcher who is not a Statistician?

{\it ``Advantages: You can gain access to data you might not have had access to before. You can also get useful guidance in what sorts of questions are important to be able to answer/what sorts of situations your method should be able to deal with, which might not seem statistically important but are very important for people who might want to use your methods.''}


{\it ``By getting a mentor from different field, a mentee can broaden the 
perspective. the different research methodology can provide a good 
insight. The tradeoff is that the mentor might not know well about main 
trends in statistics. This disadvantage gets more significant if the 
mentee is about to graduate and get a job in statistics research group.''}


% 2013  ``The mentors who are not statistican may not give useful guidance in professional development directly. However, they may inspire students to make brain storm and improve the research eciency. For example, an IOE mentors can possibly show different ways or views to a statistical problem.''

% 2014 ``Advantages: Broaden your research area. It will be easier to apply Statistics methods on other field. This will make your work more practical.

\es

\bs

  Collaboration: What are the advantages and disadvantages of building a mentorship relationship with a researcher who is not a Statistician?


{\it ``Disadvantages: Interests may conflict. Also, you have to spend extra time on understanding knowledge of other field.''}

IS THIS TIME WELL SPENT? ALSO, COMMENT ON THIS ANSWER IN THE CONTEXT OF RCRS TERMINOLOGY.

  \es \bs

% 2014 \it ``A major disadvantage arises from there being a higher likelihood of having a conflict of interest between mentors when a mentee is being advised from people in different fields.''

\es\bs

 Describe a way in which a mentorship relationship can turn unhealthy. What warning signs should one look for? What actions can one take?

{\it ``The relationship should be mutually beneficial. If one party begins relying too heavily on the other it can create problems. So if a mentor is relying too heavily on their mentee helping with their research, they may prevent the mentee from moving onto more beneficial positions. Conversely if a mentee is too heavily reliant on a mentor they may be unable to move on as well.''}



% 2014 {\it ``Often ignoring your mentor completely about an issue may lead to a rife in the mentorship relationship. It is always better to talk things out with the mentor. You do not ignore the part of your brain that says `It isn’t a good idea', you convince it that doing otherwise might suit better.''}

% 2013 ``Warning signs could involve a general feeling of dread when preparing for meetings with the mentor/mentee, ill feelings towards the other person, a signicant increase in workload, or not nishing his or her own work on time.''

% Lack of communication (solved by communicating)

% 2013 ``Unfair allocation of credit''

 \es 

%\bs

%Research collaboration is like other human relationships. Old-fashioned advice such as, ``Choose your friends carefully and then stick with them,'' is equally applicable in scientific research.

% is a human endeavor and so human interactions are important:

% 2014 \it  ``Suppose the mentor is pushing his student too hard and requires him to work long hours and work on holidays. Then their relationship can turn unhealthy. I can not come up with specific warning signs one should look for, but I believe when the relationship is starting to turn bad, one can definitely sense it. I think the wise way to not end up in such a situation is to choose your mentor carefully in first place, rather than to fix the relationship when it goes wrong. Sharing the same research interests is important, but it’s also important to choose a mentor who has similar working habits and similar values, who you get along well with as a person.''

%\es

% \bs
% 2014 \it ``A mentorship can turn unhealthy when it is no longer mutually beneficial. Often this arises when one or both parties are not meeting their responsibilities. Some warning signs of this could include lack of communication or expression of dissatisfaction. To resolve this, the mentor and mentee can revisit their goals and expectations. These can be used to form a plan to set the relationship on track. Other parties, such as a new co-mentor or graduate chair, could assist with this process.''

%\es

\end{document}
